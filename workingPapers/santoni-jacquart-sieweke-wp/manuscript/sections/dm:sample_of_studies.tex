Consistently with review articles recently published in the Journal of
Management \parencite[e.g.][]{gonzalez2018,rindova2018}, we restrict our
literature review to a selection of prominent journals such as Academy of
Management Journal, Administrative Science Quarterly, Entrepreneurship Theory
and Practice, Journal of Business Ethics, Journal of Business Venturing, Journal
of Management, Journal of Management Studies, The Leadership Quarterly,
Management Science, Organization Science, Organization Studies, Research Policy,
Strategic Entrepreneurship Journal, Strategic Management Journal, Strategic
Organization.  Using the search engine embedded in each journal's web page (see
Table \ref{tab:journal_search}) reported in the Appendix), we searched for
articles published after January 2000 that show the token ``natural
experiment*'' in the full text.

We retrieved 499 publications, 201 of which were eventually included in the
review. Figure \ref{fig:exclusion_causes}, reported in the Appendix, illustrates
the counts of excluded studies across the various categories. For example,
 We excluded 33 non-empirical publications, such as theoretical articles
\parencite[e.g.,][]{makadok2011} or review articles
\parencite[e.g.,][]{shaver2020}, and articles that: i) recall the empirical
evidence produced by previous natural experiments, ii) indicate natural
experiments as a possible way to overcome the limitations or expand on the study
at hand \parencite[e.g.,][]{hsu2006}, iii) use the instrumental variable design
\parencite[e.g.,][]{zoloty2018} or the regression discontinuity design
\parencite[e.g.,][]{flammer2015}. Finally, we excluded twelve qualitative studies
adopting the logic of natural experiment \parencite[e.g.,][]{powell2017}.

Furthermore, we filtered out articles whose authors claim to adopt a natural
experiment research design, whereas, in fact, they operate: (i) correlational
designs2 on observational data (N = 53), data produced in the context of
business simulations (N = 1), or game shows (N = 2); (ii) field experiments (N =
1); (iii) quasi-experiment/matching designs (N = 3); (iv) twin studies (N = 4).
Figure 3 reports the inter-temporal distribution of the retained studies.

