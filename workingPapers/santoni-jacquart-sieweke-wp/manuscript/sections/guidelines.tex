This section wraps up around the literature review results and provides
actionable guidelines to better leverage the natural experiment design. The
preliminary analysis of the coded data \footnote{\todo{We have already coded the
147 studies in the sample. Although further analyses are needed to reveal clear
patterns, some interesting elements seem to emerge.}} seem to indicate future
natural experiments could:

\begin{itemize}
    \item better integrate qualitative evidence and institutional knowledge in
        order to establish the as-if random nature of the treatment
    \item provide a more systematic discussion of the conditions under which a
        treatment can plausibly be considered as-if random  (see the point on 
        units' information, incentives, and capacity to self-select into the
        treatment group) 
    \item pay equal attention to the empirical and substantive relevance of the
        treatment (that is, the possibility to reveal and/or detail important 
        theoretical mechanisms by exploiting naturally-occurring events)
    \item provide a thorough assessment of the strengths and weaknesses of 
        relying on a certain naturally-occurring event --- i.e., explaining what
        the pros and cons are in terms of empirical identification (see LATE aspects)
        and theorizing opportunities 
    \item use model-based adjustments (such as matching and control covariates)
        when there is no ground to establish the (as-if) random nature of the 
        treatment. Indeed, the comparative advantage of natural experiments over
        alternative designs (e.g., quasi-experiments) also comes from the 
        possibility to conduct causal inference by means of simple, transparent 
        statical models. In other words, there should be good reasons to
        move from a design-based causal inference strategy to a model-based one
        (e.g., piggybacking on models that jointly use matching, DiD, and a long
        list of control covariates)
    \item consider the interactions among as-if random, relevance, and 
        credibility elements. For example, the credibility of a model should be 
        assessed against the nature of the treatement (random, as-if random, not
        random) and the process through which it is adminstered (see SUTVA)
\end{itemize}
