This section focuses on the diagnostics that can be used to assess and
argument the (as-if) random nature of the environmental variation at the center
of the natural experiment. Specifically, this section surveys and articulates the
following:

%\subsection{Qualitative Diagnositics}
%
%
%\subsubsection{Units' Information about the Treatment}
%
%.
%
%
%\subsubsection{Units' Incentives to Selef-Select into the Treatment Status}
%
%.
%
%
%\subsubsection{Units' Capacity to Self-Select into the Treatment Status}
%
%.
%
%
%\subsection{Quantitative Diagnostics---Balance Test}
%
%.
%

\begin{itemize}
    \item diffusion/role of qualitative diagnostics to appreciate:
        \begin{itemize}
            \item units' information about the treatment
            \item units' incentives to self-select into the treatment (control)
                group
            \item unit's capacity to self-select into the treatment (control)
                group
        \end{itemize}
    \item diffusion/role of quantitative diagnostics (e.g., balance test) to 
        compare and contrast treated and control units along relevant dimensions
\end{itemize}