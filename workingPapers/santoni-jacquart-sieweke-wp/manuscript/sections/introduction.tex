% recall causal inference issues, mention attempts made by the community
% to improve causal inference, and connect these efforts with the popularity
% of natural experiments

The quest for empirical identification in management research has created
substantial attention around `natural experiments,' a form of causal inquiry
that has been traditionally popular in economics
\parencite[][]{meyer1995,rosenzweig2000} and political science
\parencite[][]{dunning2008}.  The premise to conduct a natural experiment is the
presence of a `naturally' occurring event --- such as new regulations and laws,
natural disasters, or economic and political crises --- that heterogeneously
influences the units of a population \parencite[][]{dunning2012,robinson2009}.
Insofar as such event generates random or as-if random variations in the
environment, scholars can mimic the experimental ideal in which units are split
into a treatment and a control group or receive different levels of the
treatment. Ultimately, this opens up the possibility of inferring causal effects
when the substantive relationship at hand is difficult to investigate in a
laboratory setting and/or require operating costly, impractical, or unethical
field experiments.

% motivation for the article - NEs are not rooted in the field of strategic
% mangagement; rather, they are emerging as one of the most popular ways to
% tackle on the endogeneity challenge. As NEs diffuse in management research,
% pracrices emerge and crystallize on how to discover and leverage exogenous
% variations in order to test causal relationships. 

Although naturally occurring events can turn into opportunities to conduct
causal research, there are limited guidelines that help management scholars
prepare and review papers that implement the natural experiment research design.
To fill this gap, we highlight the strengths and weaknesses of natural
experiments as operated in the field of management studies and propose
actionable suggestions to assess and communicate the validity of natural
experiments.

To do so, we critically review the population of 147 natural experiments
published across seventeen top-tier management journals. Our review aims
to address the following research questions: \emph{R1 --- How do management
scholars claim the random or as-if random nature of environmental variation at
the core of a natural experiment? R2 --- How do they claim the empirical and
substantive relevance of a natural experiment? R3 --- How do they claim the
credibility of the statistical model encapsulated in a natural experiment
design?}

% what we do - we survey papers adopting a NE research design and take stock of
% the emerging practices. Then, we analyze these practices through a validity
% framework (i.e., Dunning's framework) in order to highlight critical
% areas/dimensions along which NE applications can be improved. Finally, we
% provide guidelines that help reviewers and authors in analyzing the validity
% of a NE paper

% contribution of the article - we aim at assisting scholars to get the most out
% of the NE paper and clarify the expectations about what a 'good' quality NE
% paper is (this could smooth the review process)


% organization of the article
This work is organized as follows. The next two sections briefly introduce the
key features of the `standard natural experiment,' \footnote{In his
comprehensive, cross-disciplinary analysis of the literature, Dunning
\citeyear[][]{dunning2012} identifies three forms of natural experiments:
Standard natural experiments; instrumental variables (Angrist, 1990); regression
discontinuity designs (Thistlethwaite \& Campbell, 1960). In the interest of
clarity and integrity, our review concentrates on standard natural experiments,
whose origin goes back to the highly acclaimed and impactful research Dr John
Snow (Snow, 1855) conducted on the diffusion of cholera in the mid
19\textsuperscript{th} century London. In this paper we use the term `natural
experiment' to exclusively refer to standard natural experiments.} along with
the evaluative framework we use to analyze the individual natural experiments.
The following section describes the selection of the reviewed studies. Then, we
present the key insights that emerge from our analysis and conclude with a
suggested checklist that helps management scholars to exploit the opportunities
of causal inference offered by naturally occurring events. The online appendix
presents an integrated set of Python scripts to implement standard natural
experiments and to assess their validity.