% -- %% ============================ Tufte' handout ===============================
% -- \documentclass[nobib]{tufte-handout}
% -- 
% -- %% ------------------------------ Tables -------------------------------------
% -- \usepackage{blkarray}          % convenient matrices
% -- \usepackage{rotating}          % rotate objects
% -- \usepackage{booktabs}          % book-quality tables
% -- \usepackage{units}             % non-stacked fractions and better unit spacing
% -- \usepackage{multicol}          % multiple column layout facilities
% -- \usepackage{lipsum}            % filler text
% -- \usepackage{fancyvrb}          % extended verbatim environments
% -- \fvset{fontsize=\normalsize} % font size for fancy-verbatim environments
% -- \usepackage{tabularx,ragged2e} % table formatting
% -- \usepackage{array}             % table formatting
% -- \usepackage{siunitx}           % alignment
% -- \sisetup{detect-all}
% -- \sisetup{input-symbols = ()}
% -- \usepackage{adjustbox}         % scaling
% -- \usepackage[T1]{fontenc}       % text
% -- \newcolumntype{P}[1]{>{\centering\arraybackslash}p{#1}}
% -- \usepackage{xcolor}            % coloured text etc.% 
% -- \usepackage{rotating}
% -- 
% -- %% --------- Standardize command font styles and environments ----------------
% -- \newcommand{\doccmd}[1]{\texttt{\textbackslash#1}}% adds backslash
% -- \newcommand{\docopt}[1]{\ensuremath{\langle}\textrm{\textit{#1}}\ensuremath{\rangle}}% optional command argument
% -- \newcommand{\docarg}[1]{\textrm{\textit{#1}}}% (required) command argument
% -- \newcommand{\docenv}[1]{\textsf{#1}}% environment name
% -- \newcommand{\docpkg}[1]{\texttt{#1}}% package name
% -- \newcommand{\doccls}[1]{\texttt{#1}}% document class name
% -- \newcommand{\docclsopt}[1]{\texttt{#1}}% document class option name
% -- \newenvironment{docspec}{\begin{quote}\noindent}{\end{quote}}% command specification environment
% -- 
% -- %% ----------------------------- Links ---------------------------------------
% -- \hypersetup{
% --     colorlinks=true,
% --     linkcolor=RoyalBlue,
% --     filecolor=RoyalBlue,      
% --     urlcolor=RoyalBlue,
% --     citecolor=Black,
% --     pdftitle={Overleaf Example},
% --     pdfpagemode=FullScreen,
% --     }
% --   
% -- %% -------------------------- Drawing and plots ------------------------------
% -- \usepackage{pgfplots}
% -- \pgfplotsset{compat=1.16}              % pgf plots
% -- \usepackage{pgf}
% -- \usepackage[utf8]{inputenc}\DeclareUnicodeCharacter{2212}{-}
% -- \usepackage{tikz}                      % tikz plots
% -- \usetikzlibrary{positioning}
% -- \usetikzlibrary{arrows.meta}
% -- \usetikzlibrary{backgrounds}
% -- \usetikzlibrary{calc}
% -- \usetikzlibrary{matrix}
% -- \usetikzlibrary{decorations.pathreplacing}
% -- \usetikzlibrary{fit}
% -- \def\firstcircle{(0,0) circle (1.5cm)} % custom shapes
% -- \def\secondcircle{(0:2cm) circle (1.5cm)}
% -- \colorlet{circle edge}{black!50}
% -- \colorlet{circle area}{black!20}
% -- \tikzset{filled/.style={fill=circle area, draw=circle edge, thick},
% -- outline/.style={draw=circle edge, thick}}
% -- \usepackage{tikz-network}
% -- \usepackage{sparklines}
% --   
% -- %% ----------------------------- Options -----------------------------------
% -- \usepackage{geometry}
% -- \renewcommand{\baselinestretch}{0.85}
% -- \usepackage{graphicx}          % allow embedded images
% --   \setkeys{Gin}{width=\linewidth, totalheight=\textheight, keepaspectratio}
% --   \graphicspath{{graphics/}}
% -- \usepackage{amsmath}                  % extended mathematics
% -- \usepackage[scaled=0.9]{helvet}       % sans-serif font to use
% -- \usepackage[defaultmathsize, symbolgreek]{mathastext}
% -- %\renewcommand\familydefault\sfdefault
% -- \Mathastext[helvet]
% -- \usepackage{sansmath}                 % maths with sans-serif
% -- \AtBeginEnvironment{tikzpicture}{\sansmath}
% -- \AtEndEnvironment{tikzpicture}{\unsansmath}
% -- \setsidenotefont{\sffamily\small}
% -- \setcaptionfont{\sffamily\small}
% -- \setmarginnotefont{\sffamily\small}
% -- \setcitationfont{\sffamily\small}
% -- 
% -- %% ----------------------------- Quotations ----------------------------------
% -- \usepackage{dirtytalk}
% -- 
% -- %% ------------------------------- Authors -----------------------------------
% -- \usepackage{authblk}
% -- 
% -- %% ----------------------------- References ----------------------------------
% -- %\usepackage[colorlinks, citecolor=DarkOrange]{hyperref}
% -- \usepackage[
% --   style=verbose,
% --   autocite=footnote,
% --   backend=biber,
% --   citestyle=authoryear
% -- ]{biblatex}
% -- \addbibresource{references/main_biblio.bib}
% -- \addbibresource{references/appendix_biblio.bib}
% -- % -------------------------- Tikz charts -------------------------------------
% -- \definecolor{base_c}{HTML}{6A94D5}
% -- \definecolor{comp_c}{HTML}{D5AB6A}
% -- \definecolor{tri_1_c}{HTML}{D56A94}
% -- \definecolor{tri_2_c}{HTML}{94D56A}
% -- \definecolor{tet_1_c}{HTML}{D56ACA}
% -- \definecolor{tet_2_c}{HTML}{D5AB6A}
% -- \definecolor{tet_3_c}{HTML}{6AD575}

%% ============================ Classic wp ===================================
\documentclass[11pt]{article}

% ------------------------------- Font ---------------------------------------
%\usepackage[T1]{fontenc}
%\usepackage{newpxtext,newpxmath}

% ------------------------- Page and paragraph layout ------------------------
\usepackage[a4paper]{geometry}
\geometry{verbose,tmargin=1in,bmargin=1in,lmargin=1in,rmargin=1in}
%\setcounter{secnumdepth}{-2}
%\setcounter{tocdepth}{-2}
\usepackage{setspace}
\onehalfspacing
%\doublespacing
\setlength{\parindent}{2.75em}
\setlength{\parskip}{1em}
%\renewcommand{\baselinestretch}{1.5}
\usepackage{titlesec}
\titlespacing{\section}{0pt}{1.5ex}{-1.5ex}
\titlespacing{\subsection}{0pt}{1.5ex}{-1.5ex}
\titlespacing{\subsubsection}{0pt}{1.5ex}{-1.5ex}
%\makeatletter
%\renewcommand{\@seccntformat}[1]{}
%\makeatother

%% -------------------------- Info on authors --------------------------------
\usepackage{authblk}

%% --------------------------- Text style ------------------------------------
\usepackage{color}

%% ----------------------------- Endnote -------------------------------------
%\usepackage{endnotes}
%\let\footnote=\endnote

%% -------------------------------- Fonts ------------------------------------
\usepackage{amsmath}                  % extended mathematics
\usepackage[scaled=0.9]{helvet}       % sans-serif font to use
\usepackage[defaultmathsize, symbolgreek]{mathastext}
%\renewcommand\familydefault\sfdefault
\Mathastext[helvet]
\usepackage{sansmath}                 % maths with sans-serif

%% ------------------------------- Links ------------------------------------- 
\usepackage[dvipsnames]{xcolor}
\usepackage[
	colorlinks=true,
	allcolors=base_c,
	%citecolor=CadetBlue,
	%urlcolor=CadetBlue
	]{hyperref}%
 
%% --------------------------- Bibliography ----------------------------------
\usepackage[
%natbib=true,
backend=biber,
style=alphabetic,
citestyle=authoryear,
%citestyle=apa,
%hyperref=true,
%maxbibnames=99,
%firstinits=false
maxcitenames=4,
%citetracker=true,
%parentracker=true,
bibstyle=authortitle
]{biblatex}

%% --------------------------- Exhibits --------------------------------------
\usepackage{array}
\usepackage{booktabs}
\usepackage{dcolumn}
\usepackage{pgf}
\usepackage{lmodern}
\usepackage{import}
\usepackage{pgfplots}
\pgfplotsset{compat=1.16}
\usepackage[utf8]{inputenc}\DeclareUnicodeCharacter{2212}{-}
\usepackage{tikz}
%\usepackage{tabularx,ragged2e}
\usepackage{threeparttable}
\usepackage{graphicx}
\usepackage{rotating}
%\usepackage[T1]{fontenc}
%\usepackage{times}
%\usepackage{amsmath}
%\usepackage{mathptmx}
%\usepackage{bbm}
%\usepackage{dsfont}
%\usepackage{etoolbox}
\usepackage{caption}
\captionsetup[table]{labelfont=sc, labelsep=newline}
\renewcommand{\figurename}{\itshape Fig.}
%\usepackage{ctable}
\renewcommand{\thetable}{\Roman{table}}
\usepackage{csquotes}
\usepackage{pdflscape}

% -------------------------- Special chars -----------------------------------
\usepackage{amssymb}

% --------------------------- Appendices -------------------------------------
\usepackage[toc, page]{appendix}

% -------------------------- Tikz charts -------------------------------------
\definecolor{base_c}{HTML}{6A94D5}
\definecolor{comp_c}{HTML}{D5AB6A}
\definecolor{tri_1_c}{HTML}{D56A94}
\definecolor{tri_2_c}{HTML}{94D56A}
\definecolor{tet_1_c}{HTML}{D56ACA}
\definecolor{tet_2_c}{HTML}{D5AB6A}
\definecolor{tet_3_c}{HTML}{6AD575}

% =============================== References ===================================
% ------------------------------ Load biblio ---------------------------------
\addbibresource{references/main_biblio.bib}
\addbibresource{references/appendix_biblio.bib}

%% =============================== Document ==================================
%% ------------------------------ Cover page ---------------------------------
%% Title
\title{Exogenous Shocks in Leadership and Management Research:\\
Types, Challenges, and Opportunities\vspace{2em}}

%% Author
\author[$\bullet$]{Simone Santoni}
\author[$\star$]{Jost Sieweke}
\author[$\circ$]{Michael Withers}
\affil[$\bullet$]{Bayes Business School --- City, University of London}
\affil[$\star$]{Vrije Universiteit Amsterdam}
\affil[$\circ$]{Mays Business School --- Texas A\&M University}

\renewcommand\Authands{ and }
%\renewcommand\Authfont{\sffamily}
\renewcommand\Affilfont{\normalsize}

%% Date
\date{\vspace{1em} \normalsize \today \vspace{1em} \\ 
      \textcolor{red}{(Structured draft --- do not circulate)}}

%% ---------------------------- Body of the document -------------------------
\begin{document}

\begin{singlespace}

\maketitle

\begin{abstract}

---Abstract goes here---

\bigskip

\textit{Keywords}: kw1, kw2, kw3.

\end{abstract}

\end{singlespace}

\clearpage

\begin{refsection}[references/main_biblio.bib]

\section{Introduction}
\label{sec:introduction}

%% recall causal inference issues, mention attempts made by the community
% to improve causal inference, and connect these efforts with the popularity
% of natural experiments

The quest for empirical identification in management research has created
substantial attention around `natural experiments,' a form of causal inquiry
that has been traditionally popular in economics
\parencite[][]{Meyer1995,Rosenzweig2000} and political science
\parencite[][]{Dunning2008}.  The premise to conduct a natural experiment is the
presence of a `naturally' occurring event --- such as new regulations and laws,
natural disasters, or economic and political crises --- that heterogeneously
influences the units of a population \parencite[][]{Dunning2012,Robinson2009}.
Insofar as such event generates random or as-if random variations in the
environment, scholars can mimic the experimental ideal in which units are split
into a treatment and a control group or receive different levels of the
treatment. Ultimately, this opens up the possibility of inferring causal effects
when the substantive relationship at hand is difficult to investigate in a
laboratory setting and/or require operating costly, impractical, or unethical
field experiments.

% motivation for the article - NEs are not rooted in the field of strategic
% mangagement; rather, they are emerging as one of the most popular ways to
% tackle on the endogeneity challenge. As NEs diffuse in management research,
% pracrices emerge and crystallize on how to discover and leverage exogenous
% variations in order to test causal relationships. 

Although naturally occurring events can turn into opportunities to conduct
causal research, there are limited guidelines that help management scholars
prepare and review papers that implement the natural experiment research design.
To fill this gap, we highlight the strengths and weaknesses of natural
experiments as operated in the field of management studies and propose
actionable suggestions to assess and communicate the validity of natural
experiments.

To do so, we critically review the population of 147 natural experiments
published across seventeen top-tier management journals.\footnote{\todo{We're
updating the literature search on May 31, 2021.}} Our review aims
to address the following research questions: \emph{R1 --- How do management
scholars claim the random or as-if random nature of environmental variation at
the core of a natural experiment? R2 --- How do they claim the empirical and
substantive relevance of a natural experiment? R3 --- How do they claim the
credibility of the statistical model encapsulated in a natural experiment
design?}

% what we do - we survey papers adopting a NE research design and take stock of
% the emerging practices. Then, we analyze these practices through a validity
% framework (i.e., Dunning's framework) in order to highlight critical
% areas/dimensions along which NE applications can be improved. Finally, we
% provide guidelines that help reviewers and authors in analyzing the validity
% of a NE paper

% contribution of the article - we aim at assisting scholars to get the most out
% of the NE paper and clarify the expectations about what a 'good' quality NE
% paper is (this could smooth the review process)


% organization of the article
This work is organized as follows. The next two sections briefly introduce the
key features of the `standard natural experiment,' \footnote{In his
comprehensive, cross-disciplinary analysis of the literature, Dunning
\citeyear[][]{Dunning2012} identifies three forms of natural experiments:
Standard natural experiments; instrumental variables (Angrist, 1990); regression
discontinuity designs (Thistlethwaite \& Campbell, 1960). In the interest of
clarity and integrity, our review concentrates on standard natural experiments,
whose origin goes back to the highly acclaimed and impactful research Dr John
Snow (Snow, 1855) conducted on the diffusion of cholera in the mid
19\textsuperscript{th} century London. In this paper we use the term `natural
experiment' to exclusively refer to standard natural experiments.} along with
the evaluative framework we use to analyze the individual natural experiments.
The following section describes the selection of the reviewed studies. Then, we
present the key insights that emerge from our analysis and conclude with a
suggested checklist that helps management scholars to exploit the opportunities
of causal inference offered by naturally occurring events. The online appendix
presents an integrated set of Python scripts to implement standard natural
experiments and to assess their validity.\footnote{\todo{Jost and I have been
working on a Python library that provides an integrated set of statistical and
visualization capabilities for the analysis of natural experiments. Happy to
share what we have produced so far.}}


.

\section{What is an exogenous shock?}
\label{sec:what_exogenous_shocks}

\subsection{Exogenous shocks in the literature on natural experiments}
\label{subsec:exogenous_shocks_and_ne}

\noindent The literature on natural experiments offers several elements to
appreciate the conceptual category of `exogenous shocks' (\cite[for an overview
of the natural experiment design, see for example][]{withers_li_2021,
dunning_2012,craig_et_al_2017,keele_et_al_2016}; \cite[for a review of the
application of this design, see][]{sekhon_titiunik_2012,sieweke_santoni_2020,
roseinzweig_et_al_2000}). Particularly, the extant works draw a line between an
exogenous shock and the interrelated but distinct concept of naturally-occurring
event. While one can observe events such as diplomatic crises, institutional
reforms, or terrorist attacks, exogenous shocks are situated in abstract models
that illustrate how economic and social formations work in the real world
\parencite{morgan_2012}. In other words, impactful naturally-occurring events,
such as Covid-19 global pandemic, could be challenging to fit within a model of
interest and, therefore, do not lead to any exogenous shock. At the same time, a
certain event could provide multiple models with an exogenous shock.  For
example, the reunification of Eastern Germany and Western Germany has been
exploited to address diverse research questions, including the impact of income
on health \parencite[e.g.][]{frijters_et_al_2004}, the transmission of
preferences for entrepreneurship from parents to children
\parencite[e.g.][]{wyrwich_2015}, or the legitimation of inequality
\parencite[e.g.][]{haack_sieweke_2018}.

Figure \ref{fig:event_rq_mapping} pictorially depicts the idea that exogenous 
shocks emerge from purposive associations of naturally-occurring events, which
create the `variance' that is necessary to test a model with observational data,
and research questions. The nature of these associations can be substantive 
--- when the environmental variation becomes an integral part of a study's 
theorizing ---, empirical --- if an analyst exploits the variation to deal with
endogeneity concerns --- or entail a combination of substantive and
empirical elements.

\begin{figure}[!htbp]
  %\sffamily
  \centering
  \includegraphics[width=0.7\textwidth]{exhibits/event_rq_mapping.pdf}
  \caption{Exogenous shocks map naturally-occurring events onto 
  research questions via empirical and substantive relevance. Notes:
  \includegraphics[width=0.075\textwidth]{exhibits/event_rq_mapping_0.pdf}
  = empirical relevance; 
  \includegraphics[width=0.075\textwidth]{exhibits/event_rq_mapping_1.pdf}
   = substantive relevance.}
  \label{fig:event_rq_mapping}
\end{figure}

%% substantive relevance
The lack of \textit{substantive relevance} is one of the most common terrains in which
naturally-occurring events are criticized. For instance, in his critical
analysis of Instrumental Variable (IV) applications,\footnote{It is commonly
accepted that natural experiments' forms include the `Instrumental Variable'
design along with the `Standard Natural Experiment' and `Regression
Discontinuity Design,'  constitutes the three forms of natural experiments
\parencite{sieweke_santoni_2020,dunning_2012}.} the prominent economist
Deaton \parencite*{deaton_2009} observes that:

\begin{quote}
\textit{
  ``[omitted] randomized evaluations of projects are useful for obtaining a convincing
  estimate of the average effect of a program or project. The price for this
  success is a focus that is too narrow to tell us `what works' in development,
  to design policy, or to advance scientific knowledge about development
  processes.''
  }
  (page 3)
\end{quote}

Hence, he argues:

\begin{quote}
   \textit{
  ``[omitted] the analysis of programs or project needs to be refocused towards
  the investigation of potentially generalizable mechanisms that explain why and
  in what contexts projects can be expected to work. The best of the
  experimental work in development economics already does so, because its
  practitioners are too talented to be bound by their own methodological
  prescriptions. Yet there would be much to be said for doing so more openly. I
  concur with the general message in Pawson and Tilley (1997), who argue that
  thirty years of project evaluation in sociology, education and criminology was
  largely unsuccessful because it focused on whether projects work instead of on
  why they work.''
  }
  (page 4)
\end{quote}

Coming from an econometric background, Heckaman and Urzua
\parencite*{heckman_urzua_2010} share Deaton's concerns about IV as a lever to
approximate the experimental ideal:

\begin{quote}
  \textit{
  ``The problem that plagues the IV approach is that the questions it answers are
  usually defined as probability limits of estimators and not by well-formulated
  economic problems. Unspecified `effects' replace clearly defined economic
  parameters as the objects of empirical interest.''
  }
  (page 28)
\end{quote}

In his comprehensive work on natural experiments, Dunning 
\parencite*{dunning_2012} pragmatically points out:

\begin{quote}
  \textit{
  ``[omitted] the causes that Nature deigns to assign at random may not always be
  the most important causal variables for social scientists. For some observers,
  the proliferation of natural experiments therefore implies the narrowing of
  research agendas to focus on substantively uninteresting or theoretically
  irrelevant topics.''
  }
  (page 3)
\end{quote}

A positive example showing how to turn a naturally-occurring event into a shock
with substantive relevance is Miguel, Satyanath, and Sergenti's
\parencite*{miguel_et_al_2004} study of the effect of economic growth on civil
war in Africa. Thanks to the ingenious choice to consider weather change data,
the authors broaden the conversation on institutions and economic growth.
Particularly, rainfall variations facilitate an IV design that copes with
longstanding endogeneity issues regarding the co-evolution of institutions and
macroeconomic factors. Hence, the authors can investigate empirically a fresh
and important theoretical relationship, namely, the effect of economic growth on
the likelihood of civil war. 

%% potential outcome framework
Regarding the \textit{empirical relevance} of a naturally-occurring event, the
literature on natural experiments shows a strict view: an event qualifies as an
exogenous shock if and only if it allows to operate a research design with
control group \parencite[][]{cook_et_al_2002}. In other words, there should be
an adequate time window\footnote{We discuss the temporal aspects of exogeneous
shocks' effects in Section \ref{sec:how_exogenous_shocks_differ} and Section
\ref{sec:harnessing_exogenous_shocks}.} in which the intervention $T$ emerging
from the exogenous shock affects a fraction a population's units only.  Such a
condition is \textit{sine qua non} to evaluate the impact of the intervention
using Neyman's potential outcome framework \parencite*{neyman_et_al_1923}, an
unbiased estimator of the average causal effect $T - C$ that is based on three
quantities: (1) $\widehat{T}$, the average response, if all subjects were assigned
to treatment; (2) $\widehat{C}$, the average response, if all subjects were assigned
to control; and (3) the difference $\hat{T} - \widehat{C}$.

The study of John Snow of London's cholera out-break of 1853-54\footnote{The
project of Snow --- aiming to prove that cholera did not transmit through the
air, the so called `miasma hypothesis'--- developed along two arms: he used
geospatial visualizations to show that attacks clustered around Broad Street
water pump in London's Soho district; then, he conducted the standard natural
experiment briefly summarized that we briefly summarize in the main text.}
emphasizes the importance of using exogenous shocks that heterogeneously affect
the units of a population. In 1852, the Lambeth Waterworks company --- one of
the major utility companies supplying water to several parts of the city ---
relocated their water works from Hungerford Market to fifteen mile upstream in
the Thames, thereby \textit{``obtaining a supply of water quite free from the
sewage of London''} \parencite[][page 68]{snow_1855}. Contrarily, Southwark \&
Vauxhall, a company competing with Lambeth Waterworks in several districts of
London, left its intake pipe downstream in the Thames at Battersea. Snow
obtained records on cholera deaths in households throughout London, as well as
information on the company that provided water to each household and the total
number of houses served by each company. Figure
\ref{fig:snow_natural_experiment} reports one of the most compelling pieces of
empirical evidence included the study: fatal attacks are compared before and
after the exogenous shock (Lambeth Waterworks moved their intake pipelines in
1952) and by water supply (see the column reported on the far-right hand section
of the table). Consistently with Neyman model, the analyses of Snow disentangle
the `true' effect of the shock on cholera communication from time invariant
attributes regarding the households served by different water suppliers, and,
perhaps more importantly, from the time variant factors such as the state of the
cholera out-break at different points in time. Should the naturally-occurring
event affected the totality of households in London, unlikely Snow's work would
have addressed the problem of confounders convincingly.  

\begin{figure}
  \begin{small}
    \begin{center}
      \includegraphics[width=0.8\textwidth]{exhibits/snow_natural_experiment.png}
    \end{center}
    \caption{An extract of the statistical analyses reported in the study of John 
    Snow `On the mode of communication of cholera' \parencite*{snow_1855}. The 
    table, reported at page 90, considers both pretest and posttest data-points ---
    Lambeth Waterworks moved their intake pipes in 1952 --- for treated 
    and control households.}
    \label{fig:snow_natural_experiment}
  \end{small}
\end{figure}

%% sudden? out of control?
A related but distinct aspect of \textit{empirical relevance} concerns the
`exogenous' nature of the naturally-occurring variation. The literature on
natural experiment emphasizes that relevant events are not necessarily sudden,
such as the death of a business leader because of hearth attack
\parencite[e.g.,][]{nguyen_et_al_2010},\footnote{The identification of sudden
deaths poses definition issues
\parencite[e.g.,][]{azoulay_et_al_2010,oettl_2012}.  In the interest of
consistency, Nguyen and Nielsen \parencite*{nguyen_et_al_2010} report they
\textit{``rely on the medical literature, which defines sudden death as an
unexpected and non-traumatic death that occurs instantaneously or within a few
hours of an abrupt change in the person’s previous clinical state.''} The causes
of sudden deaths they consider are `hearth attack,' `stroke,' and `accident or
murder.' In addition to such deaths, they consider also \textit{``accidental 
and traumatic deaths that are unanticipated by the stock market and unrelated to
firm conditions''} (page 553). } uncontrollable, such as earthquakes
\parencite[e.g.,][]{belloc_et_al_2016}, or random, such as lotteries
\parencite[e.g.][]{poulos_2019}.  In fact, extant studies show that both legal
changes and policy interventions can be purposefully used as exogenous shocks to
address certain research questions \parencite[e.g.,][]{beaman_et_al_2012,
matsa_miller_2013,chauchard_2014}. Dunning \parencite*[][page 236]{dunning_2012}
advances a three-step procedure to assess whether a naturally-occurring event
can be plausibly considered exogenous or `as-if random'. First, researchers
should investigate whether units had information that they would or would not
receive the treatment. Second, researchers need to check whether units had
incentives to self-select into the treatment group or control group. Third,
researchers should analyze whether not only units had incentives but also
capacity to self-select into a treatment status. For the assessment, Dunning
\parencite*{dunning_2012} suggests using both qualitative evidence (e.g.,
documents, interviews) and quantitative evidence (e.g., balance tests).

For example, Snow \parencite*{snow_1855} presented various sorts of evidence to
establish the pre-treatment equivalence of the houses that were exposed to pure
and contaminated sources of water supply. His own description is most eloquent:

\begin{quote}
  \textit{
  ``The mixing of the (water) supply is of the most intimate kind. The pipes of
  each Company go down all the streets, and into nearly all the courts and
  alleys. A few houses are supplied by one Company and a few by the other,
  according to the decision of the owner or occupier at that time when the Water
  Companies were in active competition. In many cases a single house has a
  supply different from that on either side. Each company supplies both rich and
  poor, both large houses and small; there is no difference either in the
  condition or occupation of the persons receiving the water of the different
  Companies\ldots It is obvious that no experiment could have been devised which
  would more thoroughly test the effect of water supply on the progress of
  cholera than this.''
  }
  \parencite[][pages 74–75]{snow_1855}
\end{quote}

At the same time, qualitative information on context and on the process that
determined water-supply source was also crucial in Snow's study. For instance,
Snow emphasized that decisions regarding which of the competing water companies
would be chosen for a particular address were often taken by absentee landlords.
Thus, residents did not largely `self-select' into their source of water
supply --- so confounding characteristics of residents appeared unlikely to
explain the large differences in death rates by company (see Figure
\ref{fig:snow_natural_experiment}). Moreover, the decision of the Lambeth
company to move its intake pipe upstream on the Thames was taken before the
cholera outbreak of 1853–54, and existing scientific knowledge did not clearly
link water source to cholera risk. As Snow puts it, the move of the Lambeth
company's water pipe meant that more than 300,000 people of all ages and social
strata were

\begin{quote}
  \textit{``divided into two groups without their choice, and, in most cases, 
  without their knowledge; one group being supplied with water containing the 
  sewage of London, and, amongst it, whatever might have come from the cholera 
  patients, the other group having water quite free from such impurity.''}
  \parencite[][pages 74–75]{snow_1855}
\end{quote}

Drawing on the methodological insights included in Snow's study, Figure 
\ref{fig:exogeneous_shocks_and_ne} illustrates visually the idea that either
events that are unknown/unknowable to units and events that are known to units
can provide scholars with an exogenous variation suited to address the research 
question at hand. However, `known events' may raise themselves endogeneity concerns
regarding the possibility for a unit to affect the direction and magnitude of a 
naturally-occurring variation and/or to self-select into the treatment 
or control group.

\begin{figure}[!htbp]
  %\sffamily
  \centering
  \includegraphics[width=1\textwidth]{exhibits/exogenous_shocks_and_ne.pdf}
  \caption{The interpretation of a naturally-occurring event as an exogenous 
  shock is contingent on the research question one wants to address and the 
  intrinsic attributes of the event. 
  Notes:  
  \includegraphics[width=0.02\textwidth]{exhibits/exogenous_shocks_and_ne_0.pdf}
  = exogenous shocks, i.e., naturally-occurring events that have empirical and/or 
  substantive relevance \textit{vis \'{a} vis} a target research question;
  \includegraphics[width=0.0175\textwidth]{exhibits/exogenous_shocks_and_ne_2.pdf}
  = naturally-occurring events are not relevant to address a target research 
  question;
  \includegraphics[width=0.0175\textwidth]{exhibits/exogenous_shocks_and_ne_1.pdf}
  = endogenous naturally-occurring events, i.e., environmental variations that 
  do not help to deal with the problem of confounders.}
  \label{fig:exogeneous_shocks_and_ne}
\end{figure}

\subsection{Exogenous shocks in leadership and management research}
\label{subsec:exogenous_shocks_in_management}

\noindent In order to understand how management scholars conceptualize and use
exogenous shocks, we conducted a systematic survey of the literature. 
Consistently with recently published reviews 
\parencite[e.g.,][]{gonzalez_et_al_2018,rindova_et_al_2018}, we restricted our
search to a selection of prominent journals such as Academy of Management
Journal, Administrative Science Quarterly, Entrepreneurship Theory and Practice,
Journal of Business Ethics, Journal of Business Venturing, Journal of
Management, Journal of Management Studies, The Leadership Quarterly, Management
Science, Organization Science, Organization Studies, Research Policy, Strategic
Entrepreneurship Journal, Strategic Management Journal, Strategic Organization.
Using Scopus, we retrieved articles published up until December 31, 2021
that present the bi-gram `exogenous shock*,' in the title, abstract, \emph{or} set of
author's generated keywords. The search resulted in 49 unique items.
\footnote{Data were retrieved on January 17, 2022.}

Having considered the full manuscript of each retrieved articles, we discarded
18 articles that did not fall within the remit of the review. Particularly, 
we excluded from the sample two studies in which the search token appears once and 
appears not to focus on or use the conceptual category of `exogenous shock'
\parencite{uzzi199735,kriauciunas2006659}; one work focusing on managers'
cognitive representation of a shock \parencite{barreto2013687}; four non-empirical papers
\parencite[e.g.,][]{mcsweeney2009933}; two qualitative studies
\parencite{glynn20051031, jenkins2010884}; one field experiment 
\parencite{cui20191216}; eight Management Science articles dealing with 
finance, marketing, or operations subjects \parencite[e.g.,][]{tham20182901}.
Figure~\ref{fig:studies_across_journals} illustrates the distribution of sample
studies across journals.

\begin{figure}[!htbp]
    \centering
    \import{}{exhibits/studies_across_journals.pgf}
    \caption{Distribution of studies that claim to use an exogenous shock across
    management journals.  Notes: N = 31; in the interest of consistency, we
    excluded `exogenous shock' studies published in Management Science addressing
    finance, marketing, or operations subjects. The following journals do not have any
    `exogenous shock' study: Administrative Science Quarterly, Entrepreneurship
    Theory and Practice, Journal of Business Venturing, Journal of Management,
    Leadership Quarterly, Research Policy.}
    \label{fig:studies_across_journals}
\end{figure}         

Two authors independently coded the retained studies against the dimensions 
included in Figure \ref{fig:event_rq_mapping}, that is, (1) the exploited 
naturally-occurring event (e.g., 9/11, Sarbane-Oxley Act); (2) the leading
research question (e.g., `how do changes in an employee's relational capital
influence mobility and entrepreneurship decisions?'); and (3) the relationship
between (1) and (2), which can be substantive --- when the exogenous shock is an
integral part of the research question/plays a key role for theorizing 
--- or empirical --- when one leverages an environmental variation to cope
with the problem of confounders, but the exogenous shock is unrelated with the
substantive scope of the study --- or both. Having completed the independent 
analysis of sample studies, the two authors merged their coding choices and 
reconciled their different views regarding the role of the exogenous shock play
in four papers.\footnote{The coding spreadsheet is publicly available at: \href{
https://www.dropbox.com/s/rzz0kb7hjoe4ec9/coded_studies.csv?dl=0}
{https://www.dropbox.com/.../coded\_studies.xlsx?dl=0}.} Table
\ref{tab:summary_of_sample_studies} reports the outcome of our coding, including
a brief summary of how the naturally-occurring event qualifies as an exogenous
shock (see the column `Summary', reported on the right-hand side of the table).

Regarding the first dimension of our coding, the most popular categories of 
events are `legal change' and (N = 11) and `turnover' (N = 6) --- see Figure 
\ref{fig:classes_of_events}. Concerning `legal change,' scholars have relied on
events such as the staggered passage of anti-takeover laws 
\parencite{cabral202128,wang20162393}, change in immigration rules 
\parencite{choudhury2019203}, the Garn-St. Germain Act
\parencite{haveman2001253}, the Sarbane-Oxley Act \parencite{gupta2020802},
SEC's regulation change \parencite{jia2020290}, the change of inheritance, gift,
and estate taxes \parencite{kang20201300}, the staggered adoption of the
Inevitable Disclosure Doctrine in U.S.  \parencite{kang20201300}, a revision of
U.S. Higher Education Amendments \parencite{krishnan20194522}, reductions in
import tariffs \parencite{li20194011}, demonetization measures
\parencite{natarajan20191070}.  Turnover events comprise sick leave episodes of
key employees \parencite{chen20181239,drexler20142722,chown2015177,}, political
leadership churn \parencite{birhanu2020,byun20191368}, and sudden deaths
of executives \parencite{ke2019439}. Other recurrent events include terrorist
attacks \parencite{corbo2016323,vergne20121027,li20194011} and scandals
\parencite{cai2019159,hilary2021}, and change in financial analysts' coverage
\parencite{chatterji2010917,qian20192271}.

\begin{figure}[!htbp]
    \centering
    \import{}{exhibits/classes_of_events.pgf}
    \caption{Classes of naturally-occurring events presumed to create 
    exogenous shocks.}
    \label{fig:classes_of_events}
\end{figure}

As shown in the Venn diagram included in Figure \ref{fig:event_roles},
naturally-occurring events are claimed to play an empirical role in the large majority
of the cases included in our sample (N = 25). For instance, Krishnan and 
Wang \parencite*{krishnan20194522} use data from the Survey of Consumer 
Finances to address the research question
`does student debt influence the propensity to start a firm?'. The concept 
of exogenous shock is not important for the scope of the work 
and does not inform the proposed theorizing. At the same time, one may study
the relationship between student debt and business creation in the context of 
an observational research design. However, the authors are concerned about 
the causal interpretation of their empirical estimates:

\begin{quote}
  \textit{
  ``there may be unobserved characteristics that may drive our results. For
  instance, individuals with wealthier families may have lower student debt as
  well as the financial means to start a firm. Such unobservable family effects
  may explain the negative relation between student debt and entrepreneurship.
  Alternatively, individuals from wealthier families may borrow more if they
  expect to be able to pay back the loans easily (and such individuals are more
  likely to be entrepreneurs).''
  }
  (page 4528)
\end{quote}

\begin{figure}[!htbp]
    \centering
    \import{}{exhibits/event_roles.pgf}
    \caption{Role of the naturally-occurring events presumed to create exogenous shocks.}
    \label{fig:event_roles}
\end{figure}

Hence, they use a legal change as an exogenous shock to the cost of business 
failures for individuals with greater levels of student loans:

\begin{quote}
  \textit{
  ``To address endogeneity concerns, we utilize the Higher Education Amendments
  (HEA) of 1998, which effectively rendered student loans completely
  non-dischargeable. We find that students that were already in four-year college
  at the time of this regulation and had significant student debt were less
  likely to start a firm. This test considers only individuals who were enrolled
  prior to the year of regulation (i.e., prior to 1998) in a four-year college.
  The idea is that, for the group of individuals who are already enrolled in
  college, the regulatory change is clearly exogenous, in the sense that it
  does not drive their choice to enter college. ''
  }
  (page 4532)
\end{quote}

Half of the studies circa (N = 16) assign a substantive role to exogenous
shocks.  For example, Haveman, Russo and Meyer \parencite*{haveman2001253} 
articulate a framework that links organizations' responses to discontinuous
industry-level, regulatory change. The theoretical section of the study maps the
phenomenon of regulatory change onto the General Punctuated Equilibrium Model
and provides expectations about the multiple consequences of regulatory change
for individual organizations. The substantive role of the shock is self-evident
in the formulation of the hypotheses, e.g.:

\begin{quote}
  \textit{
    ``Immediately following any regulatory punctuation, CEO succession rates
    will not rise; instead, CEO succession rates will rise gradually as time
    passes.''
  }
  (page 259)
\end{quote}

It is worth to notice the authors emphasize the `exogenous' nature of the Garn-St.
Germain Act --- the example of regulatory change at the center of the paper 
---, but they do not discuss what it means for the research design of 
the study, and, especially, in terms of empirical identification.

We also found a subset of studies (N = 8) that claim to use an exogenous with
empirical and substantive relevance. Byun, Raffaele, and Ganco
\parencite{byun20191368} propose a set of hypothesis that connect `discontinuous
increases in the value of an employee's relational capital' with `employee
turnover' and `spinout' formation. For example, their first hypothesis states:

\begin{quote}
  \textit{
    ``Discontinuous increases in the value of an  employee's relational capital 
    will be positively related to employee exit.''
  }
  (page 1371)
\end{quote}

The authors can advance and test this and the other hypotheses thanks to
naturally-occuring events regarding the politicians connected to lobbyists
(i.e., `employees').  Here is a passage concerning the description of the
independent variable of the study, `discontinuous increases in the value of an
employee's relational capital:'

\begin{quote}
  \textit{
    ``We use appointments to committee chair and assignments to the four most
    powerful committees in Congress to capture connected politicians' power
    changes in the legislative process. Discontinuous increase is a binary
    variable coded `1' for the first year a politician connected to a lobbyist is
    selected to be a chair of a congressional committee or is assigned to one of
    the powerful committees in Congress and `0' otherwise.''
  }
  (page 1375)
\end{quote}

The series of political appointment decisions play also an empirical role, the 
Byun and colleagues point out: 

\begin{quote}
  \textit{
    ``Our identifying assumption is consistent with prior work and rests on the
    notion that the temporal change in power of connected politicians is
    exogenous, conditional on the observable characteristics of the lobbyists
    and their firms [omitted]. For the power change of a connected politician to
    be plausibly exogenous, whether and when the connected politician will
    experience the advancement has to be difficult to predict by lobbyists and
    firms. In addition, the change in lobbyist's value creation due to a surge
    in the value of political connections should be uncorrelated with the
    accumulation of the lobbyist's expertise, conditional on observables. Given
    the complicated and uncertain political process of chair selection and
    committee assignment, scholars have argued that committee and chair
    assignment satisfies these conditions with respect to lobbyists [omitted].
    In fact, others have gone as far as to argue that the timing and ascension
    of committee and chair appointments are exogenous even to the politician
    herself [omitted]. Thus, it is reasonable to believe that using the power
    change of a connected politician to capture discontinuous in-creases in the
    lobbyist's relational capital would alleviate identification concerns due to
    potential omitted variable biases.''
  }
  (page 1375)
\end{quote}

\begin{figure}
  \raggedleft
  \begin{small}
    \import{}{exhibits/potential_outcome.pgf}
    \caption{
      Distribution of studies across research designs and exogenous shocks'
      roles. Notes. --- A design lacking a control group has pretest and posttest
      observations for one group only; a design with a control group has
      pre-test and post-test both for units that are affected by the exogenous
      shock and those that are not.
    }
    \label{fig:potential_outcome}
  \end{small}
\end{figure}

Figure \ref{fig:potential_outcome} illustrates the distribution of sample studies 
across research designs and exogenous shocks' roles. Circa one third of 
the articles (N = 11) operate a research design without control group. For
example, Corbo, Ferriani, and Corrado \parencite*{corbo2016323} use 9/11 
as a shock for the logics behind the formation of alliances in the airline industry.


% define new column type
\newcolumntype{Y}[1]{%
  >{\small\raggedright\everypar{\hangindent=1em}\arraybackslash}p{#1}%
}
% define newline to use hangindent on new line
\renewcommand{\arraystretch}{1.3}

\begin{sidewaystable}[!htbp]
  \centering
  %\sffamily
  \label{tab:summary_of_sample_studies}
  \begin{small}
    \caption{Summary of study events, research questions, and exogenous shocks}
    \vspace{-1.75em}
    \begin{center}
       %\resizebox{1\textwidth}{!}{%
       \begin{tabular}{{@{\extracolsep{2pt}} 
         p{3.85cm}@{\hskip 4mm}   %1 
         Y{4cm}@{\hskip 4mm}   %2
         Y{4cm}@{\hskip 4mm}   %3
         p{0.5cm}@{\hskip 4mm}   %4
         p{0.5cm}@{\hskip 4mm}   %5
         Y{7cm}@{\hskip 4mm} %6
         }}
         \toprule \toprule
         & %1
         & %2
         & %3
         \multicolumn{3}{l}{Relevance of the event}\\ \cmidrule{4-6}
         \multicolumn{1}{l}{Study} &
         \multicolumn{1}{l}{Event} &
         \multicolumn{1}{l}{Research question} &
         \multicolumn{1}{c}{Empirical} &
         \multicolumn{1}{c}{Substantive} &
         \multicolumn{1}{l}{Summary}\\
         \midrule \\[-1.8ex]

         Birhanu \& Wezel \parencite*{birhanu2020}\dotfill &
         Government changes following Arab spring social movement. &
         How does group affiliation influence firm performance under weak market
         institutions?&
         \multicolumn{1}{c}{$\checkmark$} &
         \multicolumn{1}{c}{$\times$} &
         The use of sudden government change is presumed to affect executives'
         capacity to influence political leaders.\\ \\[-1.8ex]

         Byun et al. \parencite*{byun20191368}\dotfill&
         Change in a politician's committee and/or committee chair assignments. &
         How do changes in an employee's relational capital influence mobility 
         and entrepreneurship decisions? &
         \multicolumn{1}{c}{$\checkmark$} &
         \multicolumn{1}{c}{$\checkmark$} &
         Lobbyists may experience a discontinuous shift in the value associated
         with a connection if there are changes to a politician's committee
         and/or committee chair assignments. Then, the authors can
         investigate empirically the consequences of social capital change on
         lobbyists' career. \\ \\[-1.8ex]
         
         Cabral et al. \parencite*{cabral202128}\dotfill&
         Staggered passage of anti-takeover laws in U.S. &
         Does managerial job security affect the adoption of innovative 
         practices and structures?&
         \multicolumn{1}{c}{$\checkmark$} &
         \multicolumn{1}{c}{$\checkmark$} &
         The adoption of an anti-takeover statute is a proxy of managerial job
         security, which changes across states and within individual states over
         time, and is supposed to affect the propensity to create a CVC
         program. \\ \\[-1.8ex]

         Cai \& Shi \parencite*{cai2019159}\dotfill &
         Revelation of the sex abuse of children by Catholic priests in U.S. &
         Does a firm's religious environment influence outside parties' 
         perceptions in contracting with the firm? &
         \multicolumn{1}{c}{$\checkmark$} &
         \multicolumn{1}{c}{$\times$} &
         Revelation of the sex abuse of children by Catholic priests is an
         exogenous shock to the religiosity of a region, which  can 
         influence the capital structure, credit rating, cost of debt, and
         covenants of local firms.\\ \\[-1.8ex]
         \bottomrule
       
        \end{tabular}
       %} 
    \end{center}
  \end{small}
\end{sidewaystable}

\begin{sidewaystable}[!htbp]
  \centering
  %\sffamily
  \begin{small}
    \caption*{\textsc{Table I} (cont'd)}
    \vspace{-1.75em}
    \label{tab:}
    \begin{center}
       %\resizebox{1\textwidth}{!}{%
       \begin{tabular}{{@{\extracolsep{2pt}}
         p{4.20cm}@{\hskip 4mm}   %1 
         Y{4cm}@{\hskip 4mm}   %2
         Y{4cm}@{\hskip 4mm}   %3
         p{0.5cm}@{\hskip 4mm}   %4
         p{0.5cm}@{\hskip 4mm}   %5
         Y{7cm}@{\hskip 4mm} %6
         }}
         \toprule \toprule
         & %1
         & %2
         & %3
         \multicolumn{3}{l}{Relevance of the event}\\ \cmidrule{4-6}
         \multicolumn{1}{l}{Study} &
         \multicolumn{1}{l}{Event} &
         \multicolumn{1}{l}{Research question} &
         \multicolumn{1}{c}{Empirical} &
         \multicolumn{1}{c}{Substantive} &
         \multicolumn{1}{l}{Summary}\\
         \midrule \\[-1.8ex]

         Chatterji \& Fabrizio \parencite*{chatterji2016447}\dotfill&
         Department of Justice investigation against the five largest U.S. 
         orthopedic device makers.&
         How does an open system of innovation affect the rate and direction 
         of innovation?&
         \multicolumn{1}{c}{$\checkmark$} &
         \multicolumn{1}{c}{$\times$} &       
         Department of Justice investigation increases the frictions in the 
         market for ideas, by regulating the interactions between physicians 
         and the medical device firms under investigation.\\
         \\[-1.8ex]
         
         Chatterji \& Toffel \parencite*{chatterji2010917}\dotfill&
         Change in the scope of KLD Database, a prominent source of CSR ratings.&
         How do managers react to poor corporate environmental ratings?&
         \multicolumn{1}{c}{$\checkmark$} &
         \multicolumn{1}{c}{$\checkmark$} &       
         The change in KLD's scope creates a subset of
         companies responding for the first time to a CSR rating, which allows
         the authors to deal with mutual causality issues regarding a firm's CSR
         rating and CSR strategy.\\ \\[-1.8ex]
         
         Chen \& Garg \parencite*{chen20181239}\dotfill &
         Injuries occurring to star NBA players. &
         Does a star's temporary absence help the organization overcome myopia?&
         \multicolumn{1}{c}{$\times$} &
         \multicolumn{1}{c}{$\checkmark$} &       
         The absence of star players is presumed to impact the pattern of
         organizational routines at the team level. \\ \\[-1.8ex]
         
         Choudhury \& Kim \parencite*{choudhury2019203}\dotfill&
         Change in U.S. H1B employment visas.&
         How do migrant inventors influence knowledge production and reuse? &
         \multicolumn{1}{c}{$\times$} &
         \multicolumn{1}{c}{$\checkmark$} &
         The H1B quota change exempted universities and a selected list of other
         entities, creating heterogeneous effects in terms of supply of first-generation
         ethnic migrant inventors and the rate of codification of knowledge
         previously locked within migrant inventors' home countries.\\ \\[-1.8ex]

         \bottomrule
        \end{tabular}
        %} 
      \end{center}
    \end{small}
  \end{sidewaystable}
  
\begin{sidewaystable}[!htbp]
    \centering
    %\sffamily
    \begin{small}
      \caption*{\textsc{Table I} (cont'd)}
      \vspace{-1.75em}
      \label{tab:}
      \begin{center}
        %\resizebox{1\textwidth}{!}{%
        \begin{tabular}{{@{\extracolsep{2pt}}
          p{3.85cm}@{\hskip 4mm}   %1 
          Y{4cm}@{\hskip 4mm}   %2
          Y{4cm}@{\hskip 4mm}   %3
          p{0.5cm}@{\hskip 4mm}   %4
          p{0.5cm}@{\hskip 4mm}   %5
          Y{7cm}@{\hskip 4mm} %6
          }}
          \toprule \toprule
          & %1
          & %2
          & %3
          \multicolumn{3}{l}{Relevance of the event}\\ \cmidrule{4-6}
          \multicolumn{1}{l}{Study} &
          \multicolumn{1}{l}{Event} &
          \multicolumn{1}{l}{Research question} &
          \multicolumn{1}{c}{Empirical} &
          \multicolumn{1}{c}{Substantive} &
          \multicolumn{1}{l}{Summary}\\
          \midrule \\[-1.8ex]

         Chown \& Liu \parencite*{chown2015177}\dotfill &
         Turnover within U.S. Senate and `iconoclastic' senators deviating from 
         the institutionalized seating arrangement. &
         How does one's location in an organizational forum affect the
         likelihood to receive support from peers?&
         \multicolumn{1}{c}{$\checkmark$} &
         \multicolumn{1}{c}{$\times$} &
         Turnover within U.S. Senate and `iconoclastic' senior senators create
         opportunities for freshman senators not to seat at the margins of the
         chamber. These elements affect the dyadic distance between senators, a factor
         that is presumed to affect the likelihood of joint support. \\ \\[-1.8ex]

         Corbo et al. \parencite*{corbo2016323}\dotfill&
         9/11.&
         Does a major environmental shock affect the social structure of an 
         organizational field?&
         \multicolumn{1}{c}{$\times$} &
         \multicolumn{1}{c}{$\checkmark$} &
         9/11 is supposed to affect the organization and functioning of civil
         aviation, which allows the authors to assess the extent with which
         network mechanisms shape the alliances connecting airline companies 
         under different contingencies.\\ \\[-1.8ex]

         Drexler \& Schoar \parencite*{drexler20142722}\dotfill &
         Sick leave episodes among loan officers &
         How (much) does employee turnover affect organizational performance? &
         \multicolumn{1}{c}{$\checkmark$} &
         \multicolumn{1}{c}{$\checkmark$} &
         Loan officers' sick leaves alter economic and social exchange between
         the firm and its clients.\\ \\[-1.8ex]
        
         Gupta et al. \parencite*{gupta2020802}\dotfill &
         Sarbanes-Oxley Act (SOX) \& Global Financial Crisis &
         Does CFO gender influence the likelihood of financial misreporting? &
         \multicolumn{1}{c}{$\checkmark$} &
         \multicolumn{1}{c}{$\times$} &
         The authors expect: i) SOX to lead to a larger decrease in financial
         misreporting for male CFO firms than female-CFO firm; ii) firms to face
         greater pressure to report favorable earnings during crisis periods,
         which is more likely to influence male compared to female CFOs (based
         on the logic that female CFOs will be less likely to engage in fraud
         regardless of stakeholder pressure).\\  \\[-1.8ex]
          
         \bottomrule
       \end{tabular}
       %} 
    \end{center}
  \end{small}
\end{sidewaystable}

\begin{sidewaystable}[!htbp]
  \centering
  %\sffamily
  \begin{small}
    \caption*{\textsc{Table I} (cont'd)}
    \vspace{-1.75em}
    \label{tab:}
    \begin{center}
       %\resizebox{1\textwidth}{!}{%
       \begin{tabular}{{@{\extracolsep{2pt}}
         p{3.85cm}@{\hskip 4mm}   %1 
         Y{4cm}@{\hskip 4mm}   %2
         Y{4cm}@{\hskip 4mm}   %3
         p{0.5cm}@{\hskip 4mm}   %4
         p{0.5cm}@{\hskip 4mm}   %5
         Y{7cm}@{\hskip 4mm} %6
         }}
         \toprule \toprule
         & %1
         & %2
         & %3
         \multicolumn{3}{l}{Relevance of the event}\\ \cmidrule{4-6}
         \multicolumn{1}{l}{Study} &
         \multicolumn{1}{l}{Event} &
         \multicolumn{1}{l}{Research question} &
         \multicolumn{1}{c}{Empirical} &
         \multicolumn{1}{c}{Substantive} &
         \multicolumn{1}{l}{Summary}\\
         \midrule \\[-1.8ex]

         Haveman et al. \parencite*{byun20191368}\dotfill&
         California Legislature enactment of the nation's first comprehensive 
         managed competition program, and Garn-St. Germain Act&
         How do organizations respond to discontinuous indsutry-level change?&
         \multicolumn{1}{c}{$\times$} & 
         \multicolumn{1}{c}{$\checkmark$} &
         The authors use a series of regulatory changes to investigate how
         organizations respond to punctuated changes in the environment and with
         what performance consequences.\\ \\[-1.8ex]
          
         Hilary \& Huang \parencite*{hilary2021}\dotfill &
         Revelation of the sex abuse of children by Catholic priests in U.S. &
         Does generalized trust affect the power of CEO contracts? & 
         \multicolumn{1}{c}{$\checkmark$} & 
         \multicolumn{1}{c}{$\times$} &
         Revelation of the sex abuse of children by Catholic priests
         reduces generalized trust for certain counties only, which helps to reveal
         the causal effect of generalized trust on the characteristics of
         executives' contracts.\\ \\[1.8ex] 

         Jia et al. \parencite*{jia2020290}\dotfill&
         2005 Regulation SHO by which SEC removes the uptick restriction for a
         set of randomly selected pilot firms.&
         Do managers use CSR to insure against stock price risk?&
         \multicolumn{1}{c}{$\checkmark$} & 
         \multicolumn{1}{c}{$\checkmark$} &
         SEC program changes stock risk price for pilot firms only, which helps
         to assess whether firms invest in CSR in response to greater stock
         price risk, and whether such investments provide intended
         insurance-like benefits.\\ \\[-1.8ex]

         Kang \& Kim \parencite*{kang20201300}\dotfill &
         Staggered changes in inheritance, gift, and estate 
         taxes in U.S. \& sudden deaths of business owners. &
         Do family-firms invest more in employee relations than 
         non-family firms? & 
         \multicolumn{1}{c}{$\checkmark$} & 
         \multicolumn{1}{c}{$\times$} &
         Taxation changes provide family owners with
         incentives to continue their businesses, which helps to
         reveal the relationship between governance forms and investment in
         employee relations.

         Sudden death of family members alter a firm's status, a variation 
         that attenuates the concerns time-invariant characteristics jointly 
         affect performance and ability to implement employee-friendly 
         policies.\\ \\[-1.8ex]

         \bottomrule
       \end{tabular}
       %} 
    \end{center}
  \end{small}
\end{sidewaystable}

\begin{sidewaystable}[!htbp]
  \centering
  %\sffamily
  \begin{small}
    \caption*{\textsc{Table I} (cont'd)}
    \vspace{-1.75em}
    \label{tab:}
    \begin{center}
       %\resizebox{1\textwidth}{!}{%
       \begin{tabular}{{@{\extracolsep{2pt}}
         p{3.85cm}@{\hskip 4mm}   %1 
         Y{4cm}@{\hskip 4mm}   %2
         Y{4cm}@{\hskip 4mm}   %3
         p{0.5cm}@{\hskip 4mm}   %4
         p{0.5cm}@{\hskip 4mm}   %5
         Y{7cm}@{\hskip 4mm} %6
         }}
         \toprule \toprule
         & %1
         & %2
         & %3
         \multicolumn{3}{l}{Relevance of the event}\\ \cmidrule{4-6}
         \multicolumn{1}{l}{Study} &
         \multicolumn{1}{l}{Event} &
         \multicolumn{1}{l}{Research question} &
         \multicolumn{1}{c}{Empirical} &
         \multicolumn{1}{c}{Substantive} &
         \multicolumn{1}{l}{Summary}\\
         \midrule \\[-1.8ex]

         Ke et al. \parencite*{ke2019439}\dotfill &
         Sudden deaths and retirements of executives. &
         How do social connections among executive team members affect 
         management forecast accuracy?&
         \multicolumn{1}{c}{$\checkmark$} & 
         \multicolumn{1}{c}{$\times$} &
         Sudden turnover events alter the social connections within a team of
         executives, and, in turn, help to reveal the causal effect of social
         capital on decision-making quality. \\ \\[-1.8ex]
         
         Koh et al. \parencite*{koh20185725}\dotfill &
         Staggered adoption of the Inevitable Disclosure Doctrine (IDD).
         in U.S. &
         Are confident CEOs more likely to report R\&D expenditures than
         cautious CEOs? &
         \multicolumn{1}{c}{$\checkmark$} & 
         \multicolumn{1}{c}{$\times$} &
         The staggered U.S. state courts' verdict on the IDD helps to
         reveal the relationship between CEO confidence and R\&D disclosure by
         attenuating market competition.\\ \\[-1.8ex] 

         Krishnan \& Wang \parencite*{krishnan20194522}\dotfill&
         1992 and 1998 Higher Education Amendments (HEA) &
         How does student debt influence the propensity to start a firm? &
         \multicolumn{1}{c}{$\checkmark$} & 
         \multicolumn{1}{c}{$\times$} &
         1998 HEA alters the cost of discharging student debt through bankruptcy
         --- which increases the cost of entrepreneurship, that is, new venture
         failure --- while it is unlikely to affect financing availability to start
         a venture. Hence, the authors can assess the causal relationship
         linking student debt with propensity to create a new venture.
         
         1992 HEA affects the volume of student loans thought the federal
         government.  Students who spend more time in college during the
         post-1992 HEA regime will have more student loans. Hence, they will
         have lower likelihood to start a new venture. \\ \\[-1.8ex]

         Li \& Tallman \parencite*{li20111119}\dotfill&
         9/11.&
         Does a sudden change in the environment influence the economic 
         returns of international diversification?&
         \multicolumn{1}{c}{$\times$} & 
         \multicolumn{1}{c}{$\checkmark$} &
         9/11 is a ``reorienting disruptive change'' that alters international
         business logics, particularly, the economic and finial returns of
         international diversification.\\ \\[-1.8ex]
         
         \bottomrule
       \end{tabular}
       %} 
    \end{center}
  \end{small}
\end{sidewaystable}

\begin{sidewaystable}[!htbp]
  \centering
  %\sffamily
  \begin{small}
    \caption*{\textsc{Table I} (cont'd)}
    \vspace{-1.75em}
    \label{tab:}
    \begin{center}
       %\resizebox{1\textwidth}{!}{%
       \begin{tabular}{{@{\extracolsep{2pt}}
         p{3.85cm}@{\hskip 4mm}   %1 
         Y{4cm}@{\hskip 4mm}   %2
         Y{4cm}@{\hskip 4mm}   %3
         p{0.5cm}@{\hskip 4mm}   %4
         p{0.5cm}@{\hskip 4mm}   %5
         Y{7cm}@{\hskip 4mm} %6
         }}
         \toprule \toprule
         & %1
         & %2
         & %3
         \multicolumn{3}{l}{Relevance of the event}\\ \cmidrule{4-6}
         \multicolumn{1}{l}{Study} &
         \multicolumn{1}{l}{Event} &
         \multicolumn{1}{l}{Research question} &
         \multicolumn{1}{c}{Empirical} &
         \multicolumn{1}{c}{Substantive} &
         \multicolumn{1}{l}{Summary}\\
         \midrule \\[-1.8ex]

         Li \& Zhan \parencite*{li20194011}\dotfill&
         Reductions in import tariffs initiated by U.S. authorities. &
         How does product market threats affect stock crash risk?&
         \multicolumn{1}{c}{$\checkmark$} & 
         \multicolumn{1}{c}{$\times$} &
         Reduction in import tariffs increases competitive pressure, which 
         aggravates executives' incentive to withhold negative information 
         and increases and make firms more prone to stock crashes.\\ \\[-1.8ex]

         Mahmood et al. \parencite*{mahmood20171082} \dotfill &
         Global Financial Crisis.&
         How does centralization of intragroup equity ties affects the 
         performance of group affiliates?&
         \multicolumn{1}{c}{$\checkmark$} & 
         \multicolumn{1}{c}{$\checkmark$} &
         Global Financial Crisis creates environmental turbulence exogenously
         for Taiwanese firms, and, in turn, helps to appreciate the
         contingent role of equity tie centralization.\\ \\[-1.8ex]

         Natarjan et al. \parencite*{natarajan20191070}\dotfill&
         Indian Government's demonetization measure.&
         How do middle managers influence resource allocation choices?&
         \multicolumn{1}{c}{$\checkmark$} & 
         \multicolumn{1}{c}{$\times$} &
         The decision to withdraw almost 85\% of bank notes in circulation (all
         500-rupee and 1,000-rupee bills, the most common units of circulating
         currency) increased bank headquarters' control over ATM deployment,
         which resulted in tighter monitoring of middle managers' allocation
         decisions.\\ \\[-1.8ex]

         Qian et al. \parencite*{qian20192271}\dotfill &
         Brokerage house mergers and closures in U.S. &
         How do financial analysts influence managers' choice to invest in 
         CSR?&
         \multicolumn{1}{c}{$\checkmark$} & 
         \multicolumn{1}{c}{$\checkmark$} &
         The closure or merger regarding a brokerage house reduces financial
         analyst coverage for some firms only, which allow the authors to assess
         the causal relationship between the (change in the) extent of analyst
         coverage and CSR.\\ \\[-1.8ex]

         \bottomrule
       \end{tabular}
       %} 
    \end{center}
  \end{small}
\end{sidewaystable}

\begin{sidewaystable}[!htbp]
  \centering
  %\sffamily
  \begin{small}
    \caption*{\textsc{Table I} (cont'd)}
    \vspace{-1.75em}
    \label{tab:}
    \begin{center}
       %\resizebox{1\textwidth}{!}{%
       \begin{tabular}{{@{\extracolsep{2pt}}
         p{3.85cm}@{\hskip 4mm}   %1 
         Y{4cm}@{\hskip 4mm}   %2
         Y{4cm}@{\hskip 4mm}   %3
         p{0.5cm}@{\hskip 4mm}   %4
         p{0.5cm}@{\hskip 4mm}   %5
         Y{7cm}@{\hskip 4mm} %6
         }}
         \toprule \toprule
         & %1
         & %2
         & %3
         \multicolumn{3}{l}{Relevance of the event}\\ \cmidrule{4-6}
         \multicolumn{1}{l}{Study} &
         \multicolumn{1}{l}{Event} &
         \multicolumn{1}{l}{Research question} &
         \multicolumn{1}{c}{Empirical} &
         \multicolumn{1}{c}{Substantive} &
         \multicolumn{1}{l}{Summary}\\
         \midrule \\[-1.8ex]

         Ramirez \& \parencite*{ramírez20181496}\dotfill&
         Price variation in the global copper industry.&
         How does the value appropriated by employees varies in response to 
         an exogenous shock to the price of the firm's product?&
         \multicolumn{1}{c}{$\checkmark$} & 
         \multicolumn{1}{c}{$\checkmark$} &
         Copper mines' size is homogeneous. Hence, price fluctuations in the
         global copper industry are exogenous variations for individual mines
         and can reveal the mechanisms behind value distribution within 
         organizations.\\ \\[-1.8ex] 

         Seebeck \& Vetter \parencite*{seebeck2021}\dotfill &
         Brexit Referendum. &
         Does board gender diversity affect corporate risk disclosure? &
         \multicolumn{1}{c}{$\times$} &
         \multicolumn{1}{c}{$\checkmark$} &
         The outcome of Brexit Referendum increases the amount of risk
         environment for all UK-based companies, which attenuates reverse
         causality concerns regarding board gender diversity on corporate and
         risk disclosure. \\ \\[-1.8ex]

         Tan \& Netessine \parencite*{tan20141574}\dotfill&
         Adoption of a new staffing system.&
         How does workload impact worker productivity? &
         \multicolumn{1}{c}{$\checkmark$} &
         \multicolumn{1}{c}{$\times$} &
         The staggered adoption of a new computer-based scheduling system
         prescribes different staffing levels from those that managers might
         suggest because it uses more historical sales data than a manager can
         handle. Hence, authors can make cross-restaurant comparisons involving
         similar servers experiencing different workload levels.\\  \\[-1.8ex]

         \bottomrule
       \end{tabular}
       %} 
    \end{center}
  \end{small}
\end{sidewaystable}


\begin{sidewaystable}[!htbp]
  \centering
  %\sffamily
  \begin{small}
    \caption*{\textsc{Table I} (cont'd)}
    \vspace{-1.75em}
    \label{tab:}
    \begin{center}
       %\resizebox{1\textwidth}{!}{%
       \begin{tabular}{{@{\extracolsep{2pt}}
         p{3.85cm}@{\hskip 4mm}   %1 
         Y{4cm}@{\hskip 4mm}   %2
         Y{4cm}@{\hskip 4mm}   %3
         p{0.5cm}@{\hskip 4mm}   %4
         p{0.5cm}@{\hskip 4mm}   %5
         Y{7cm}@{\hskip 4mm} %6
         }}
         \toprule \toprule
         & %1
         & %2
         & %3
         \multicolumn{3}{l}{Relevance of the event}\\ \cmidrule{4-6}
         \multicolumn{1}{l}{Study} &
         \multicolumn{1}{l}{Event} &
         \multicolumn{1}{l}{Research question} &
         \multicolumn{1}{c}{Empirical} &
         \multicolumn{1}{c}{Substantive} &
         \multicolumn{1}{l}{Summary}\\
         \midrule \\[-1.8ex]

         Vergne \parencite*{vergne20121027}\dotfill &
         9/11. &
         Does straddling multiple product-market categories dilute stakeholder 
         attention to the stigma of operating in the global army industry?&
         \multicolumn{1}{c}{$\times$} &
         \multicolumn{1}{c}{$\checkmark$} &
         Since attackers used commercial airlines hijacked by terrorists
         armed with kitchen knives, the definition of the weapons category was
         questioned in the post-9/11 period.'' Hence, 9/11 allows the author to
         test whether the salience of the category `weapons' weakens `the negative
         relationship between stigma dilution (i.e., the situation in which a
         diversified business operates also in a stigmatized sector, such as
         `arms') and media disapproval.\\ \\[-1.8ex]
       
         Wang et al. \parencite*{wang20162393}\dotfill&
         Delaware's 1996 ruling against hostile takeovers. &
         Do takeover threats affect a firm's knowledge structure?&
         \multicolumn{1}{c}{$\checkmark$} &
         \multicolumn{1}{c}{$\times$} &
         A series of law cases make takeover less favorable for target firms
         incorporated in Delaware, allowing the authors to assess the impact of
         (an increase in) takeover protection on firm-level knowledge
         production.\\ \\[-1.8ex]

         Zhang et al. \parencite*{zhang2020}\dotfill &
         iOS 7 jailbreak. &
         Does a lapse in gatekeeping reduces knowledge sharing among 
         developers?&
         \multicolumn{1}{c}{$\checkmark$} &
         \multicolumn{1}{c}{$\checkmark$} &
         The event is an exogenous shock to Apple's gatekeeping policy, aiming
         to orchestrate developers' value creation activities in the AppStore.
         That allows scholars to appreciate the impact of platform governance on
         knowledge sharing among developers.\\ \\[-1.8ex]

         Zheng \& Wang \parencite*{zheng20202234}\dotfill &
         2014 Google blockade in China. &
         How does Google's search engine influence the search process of 
         inventors?&
         \multicolumn{1}{c}{$\times$} &
         \multicolumn{1}{c}{$\checkmark$} &
         The blockade of Google affects inventor's information processing and,
         in turn, innovation output.\\ \\[-1.8ex]
         \bottomrule
       \end{tabular}
       %} 
    \end{center}
  \end{small}
\end{sidewaystable}

\clearpage

\section{How do exogenous shock differ?}
\label{sec:how_exogenous_shocks_differ}

.

\begin{figure}[!htbp]
    \begin{center}
      \includegraphics[width=1\textwidth]{exhibits/typology.pdf}
    \end{center}
    \caption{A typology of exogenous shocks.}
    \label{fig:typology}
\end{figure}


\section{Harnessing exogenous shocks: opportunities and challenges}
\label{sec:harnessing_exogenous_shocks}

\begin{itemize}
\item Environmental change - theory mapping
\item Temporal effects / timeline within which one expects the effects of the exogenous shock takes place
\item Scope to influence regulator's choices?
\item Time to react to regulator's choices?
\item Does a staggered adoption of legislative change correlate with admin/state-level features?
\item Exogenous shocks may have consequences that span multiple levels of analysis (e.g., sudden deaths of business owners)
\item Strategic purposes might lead the exogenous shock — entities that are treated might be passive — however, the policy maker may have incentives to treat certain entities
\item Conditional exogeneity (exogeneity within a statistical model sounds weird)
\item The nature of regulatory changes should be investigated from a multi-level perspective
\item Opportunities to marry the theoretical and empirical approaches to the study of shocks
\item Regulatory change — information, incentives, and capacity elements should be assessed before and after the introduction of the new law.
\item Scale of the shock and availabilities of counterfactualsThe
\end{itemize}

\begin{figure}[!htbp]
  \includegraphics[width=1\textwidth]{exhibits/place_holder.pdf}
  \caption{A decision tree to harnessing exogenous shocks.}
  \label{fig:harnessing_exogeneous_shocks}
\end{figure}

\section{Coda}
\label{sec:coda}
.

\clearpage

%% --------------------------- Bibliography ---------------------------------
%\bibliographystyle{plainnat}
\section*{References}
\printbibliography[heading=none]
\end{refsection}

% -------------------------- Appendix --------------------------------------
%\begin{refsection}[references/sample.bib]
%
%\section{Appendix A --- Sample of studies}
%\label{sec:sampling}
%
%\setcounter{table}{0}
%\renewcommand{\thetable}{A\arabic{table}}
%\renewcommand{\thefigure}{A\arabic{figure}}
%
%!! Describe literature search !!
%
%\subsection{References}
%
%\printbibliography[heading=none]
%
%\end{refsection}

% ----------------------------- Closing ------------------------------------
\end{document}
